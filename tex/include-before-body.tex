%:::% class attribute begin/end %:::%

% -----
% Cover (mandatory)
% -----

%:::% cover begin %:::%
\imprimircapa
%:::% cover end %:::%

% -----
% Title page (mandatory)
% -----

%:::% approval-sheet begin %:::%
\imprimirfolhaderosto
%:::% approval-sheet end %:::%

% -----
% Cataloging record (mandatory)
% -----

%:::% cataloging-record begin %:::%
\begin{fichacatalografica}
%:::% cataloging-record body begin %:::%
I authorize the full or partial reproduction of this work by any conventional or electronic means for the purposes of study and research, provided that the source is cited.
\ABNTEXfontereduzida

\vfill
\begin{center}
Cataloging in publication

Library

{\imprimirescola}

\medskip
\setlength{\fboxsep}{1cm}
\fbox{\begin{minipage}[c][7.5cm]{12.5cm}
Sales, Alícia Rafaelly Vilefort

\smallskip
\hspace{0.5cm} {\imprimirtitulo}  / {\imprimirautor}; supervisor, {\imprimirorientador}. {\imprimirlocal}, {\imprimirdata}.

\smallskip
{\thelastpage}p. : il

\smallskip
\hspace{0.5cm} {\imprimirtipotrabalho} (\imprimirtituloacademico) -- {\imprimirprograma}, {\imprimirescola}, {\imprimiruniversidade}, {\imprimirlocal}, {\imprimirdata}.

\smallskip
\hspace{0.5cm} {\imprimirnotadeversao}

\smallskip
\hspace{0.5cm} 1. Pregnancy. 2. Sleep. 3. Childbirth. 4. Chronobiology. 5. Midwifery. I. Chofakian, Christiane Borges do Nascimento, super. II. Title.

\smallskip
\hfill CDD ? .ed. ???.???
\end{minipage}}
\end{center}
%:::% cataloging-record body end %:::%
\end{fichacatalografica}
%:::% cataloging-record end %:::%

% -----
% Errata (optional)
% -----

%:::% errata begin %:::%
\begin{errata}[\errataname]
%:::% errata reference begin %:::%
\noindent Sales, A. R. V. ({\imprimirdata}). \textit{\imprimirtitulo} [{\imprimirtipodetituloacademico}'s {\imprimirtipotrabalho}, {\imprimiruniversidade}].
%:::% errata reference end %:::%
\smallskip
%:::% errata body begin %:::%

This is the preliminary version of this thesis (version \textless1.0.0).
Any required corrections will be listed here upon approval.

%:::% errata body end %:::%
\end{errata}
%:::% errata end %:::%

% -----
% Approval sheet (mandatory)
% -----

%:::% approval-sheet begin %:::%
\begin{folhadeaprovacao}[\folhadeaprovacaoname]
%:::% approval-sheet body begin %:::%
{\imprimirtipotrabalho} by {\imprimirautor}, under the title \textbf{\imprimirtitulo}, presented to the {\imprimirescola} at the {\imprimiruniversidade}, as a requirement for the degree of {\imprimirtituloacademico} by the {\imprimirprograma}, in the concentration area of {\imprimirareadeconcentracao}.

\vspace{\hugeskipamount}
Approved on \_\_\_\_\_\_\_\_\_\_\_\_\_\_\_\_\_\_\_\_ , \_\_\_\_\_\_\_\_\_\_ .

\vspace{\hugeskipamount}
\begin{center}
  Examination committee
\end{center}

\vspace{\smallskipamount}
Committee chair:

\vspace{\tinyskipamount}
\begingroup

\AtBeginEnvironment{tabular}{
  \normalsize
  \renewcommand{\arraystretch}{2}
}

\setlength{\arrayrulewidth}{0pt}
\setlength{\tabcolsep}{0cm}
\begin{tabular}{m{2cm} P{14cm}}
  Prof. Dr. & \_\_\_\_\_\_\_\_\_\_\_\_\_\_\_\_\_\_\_\_\_\_\_\_\_\_\_\_\_\_\_\_\_\_\_\_\_\_\_\_\_\_\_\_\_\_\_\_\_\_\_\_\_\_\_ \\
  Institution & \_\_\_\_\_\_\_\_\_\_\_\_\_\_\_\_\_\_\_\_\_\_\_\_\_\_\_\_\_\_\_\_\_\_\_\_\_\_\_\_\_\_\_\_\_\_\_\_\_\_\_\_\_\_\_ \\
\end{tabular}

\vspace{\bigskipamount}
Examiners:

\vspace{\tinyskipamount}
\begin{tabular}{m{2cm} P{14cm}}
  Prof. Dr. & \_\_\_\_\_\_\_\_\_\_\_\_\_\_\_\_\_\_\_\_\_\_\_\_\_\_\_\_\_\_\_\_\_\_\_\_\_\_\_\_\_\_\_\_\_\_\_\_\_\_\_\_\_\_\_ \\
  Institution & \_\_\_\_\_\_\_\_\_\_\_\_\_\_\_\_\_\_\_\_\_\_\_\_\_\_\_\_\_\_\_\_\_\_\_\_\_\_\_\_\_\_\_\_\_\_\_\_\_\_\_\_\_\_\_ \\
  Evaluation & \_\_\_\_\_\_\_\_\_\_\_\_\_\_\_\_\_\_\_\_\_\_\_\_\_\_\_\_\_\_\_\_\_\_\_\_\_\_\_\_\_\_\_\_\_\_\_\_\_\_\_\_\_\_\_ \\
\end{tabular}

\vspace{\smallskipamount}
\begin{tabular}{m{2cm} P{14cm}}
  Prof. Dr. & \_\_\_\_\_\_\_\_\_\_\_\_\_\_\_\_\_\_\_\_\_\_\_\_\_\_\_\_\_\_\_\_\_\_\_\_\_\_\_\_\_\_\_\_\_\_\_\_\_\_\_\_\_\_\_ \\
  Institution & \_\_\_\_\_\_\_\_\_\_\_\_\_\_\_\_\_\_\_\_\_\_\_\_\_\_\_\_\_\_\_\_\_\_\_\_\_\_\_\_\_\_\_\_\_\_\_\_\_\_\_\_\_\_\_ \\
  Evaluation & \_\_\_\_\_\_\_\_\_\_\_\_\_\_\_\_\_\_\_\_\_\_\_\_\_\_\_\_\_\_\_\_\_\_\_\_\_\_\_\_\_\_\_\_\_\_\_\_\_\_\_\_\_\_\_ \\
\end{tabular}

\vspace{\smallskipamount}
\begin{tabular}{m{2cm} P{14cm}}
  Prof. Dr. & \_\_\_\_\_\_\_\_\_\_\_\_\_\_\_\_\_\_\_\_\_\_\_\_\_\_\_\_\_\_\_\_\_\_\_\_\_\_\_\_\_\_\_\_\_\_\_\_\_\_\_\_\_\_\_ \\
  Institution & \_\_\_\_\_\_\_\_\_\_\_\_\_\_\_\_\_\_\_\_\_\_\_\_\_\_\_\_\_\_\_\_\_\_\_\_\_\_\_\_\_\_\_\_\_\_\_\_\_\_\_\_\_\_\_ \\
  Evaluation & \_\_\_\_\_\_\_\_\_\_\_\_\_\_\_\_\_\_\_\_\_\_\_\_\_\_\_\_\_\_\_\_\_\_\_\_\_\_\_\_\_\_\_\_\_\_\_\_\_\_\_\_\_\_\_ \\
\end{tabular}
\endgroup
%:::% approval-sheet body end %:::%
\end{folhadeaprovacao}
%:::% approval-sheet end %:::%

% -----
% Inscription (optional)
% -----

%:::% inscription begin %:::%
\begin{dedicatoria}[] % \dedicatorianame | Keep #1 empty.
\vspace*{\fill} % Don't change it.
\centering
%:::% inscription body begin %:::%
\textit{I dedicate this work to ...}
%:::% inscription body end %:::%
\vspace*{\fill} % Don't change it.
\vspace{4.5cm}
% \vspace{13cm}
\end{dedicatoria}
%:::% inscription end %:::%

% -----
% Acknowledgments (optional)
% -----

%:::% acknowledgments begin %:::%
\begin{agradecimentos}[\agradecimentosname]
  %:::% acknowledgments body begin %:::%

I would like to acknowledge and express my gratitude to the following
persons and organizations:

The \href{https://prip.usp.br/apoio-estudantil/}{Support Program for
Student Permanence and Education (PAPFE)} of USP, which enabled me to
get this far.

The \href{https://www.gov.br/capes/}{Coordination for the Improvement of
Higher Education Personnel (CAPES)}, for funding this work and enabling
my presence in graduate studies.

  %:::% acknowledgments body end %:::%
\end{agradecimentos}
%:::% acknowledgments end %:::%

% -----
% Epigraph (optional)
% -----

%:::% epigraph begin %:::%
\begin{epigrafe}[] % \epigraphname | Keep #1 empty.
\vspace*{\fill} % Don't change it.
\begin{flushright}
%:::% epigraph body begin %:::%
\textit{Nullius in verba}\footnotemark{}

\footnotetext{
  The Royal Society. (n.d.). \textit{History of the Royal Society}. \href{https://royalsociety.org/about-us/history/}{https://royalsociety.org/about-us/history/}
}
%:::% epigraph body end %:::%
\end{flushright}
\end{epigrafe}
%:::% epigraph end %:::%

% -----
% Abstract in the vernacular language (mandatory)
% -----

%:::% vernacular-abstract begin %:::%
\begin{resumoenv}[\resumoname]
 %:::% vernacular-abstract reference begin %:::%
Sales, A. R. V. ({\imprimirdata}). \textit{\imprimirtitulo} [{\imprimirtipodetituloacademico}'s {\imprimirtipotrabalho}, {\imprimiruniversidade}].
%:::% vernacular-abstract reference end %:::%

%:::% vernacular-abstract body begin %:::%

The text below is related to the \textbf{project} of this thesis. The
final abstract can only be produced when the research is completed.

Among the biopsychosocial changes that occur during pregnancy are
alterations in the sleep-wake cycle pattern. Research suggests
associations between the duration and quality of sleep in pregnant women
during the prenatal period and adverse outcomes in maternal and infant
health. The primary objective of this project is to investigate the
presence/absence of significant associations between the duration and
quality of sleep in pregnant women during the third trimester and the
duration of labor. For this purpose, a study will be conducted involving
133 pregnant women in the third trimester, followed at the Sapopemba
birthing center located in the municipality of São Paulo, Brazil.
Demographic, anthropometric, obstetric, actigraphic, sleep-related, and
psychological state data of the participants will be collected, along
with secondary data from the pregnant women's records. The study has
already obtained all ethical approvals for its operation from the
competent authorities. The results will be analyzed by comparing the
means between groups through an analysis of covariance (ANCOVA). The
basic hypothesis is that lower sleep quality and duration throughout
gestation are associated with longer labor duration. In addition to
generating knowledge on a matter of public interest, it is expected that
the results of this project will encourage and contribute to the
creation of new services and technologies for monitoring pregnant women.

%:::% vernacular-abstract body end %:::%

%:::% vernacular-abstract keywords begin %:::%
\begin{tabular}{p{2.3cm} p{13.6cm}}
  \textbf{Keywords}: & Pregnancy. Sleep. Sleep-wake cycle. Sleep quality. Duration of sleep. Childbirth. Duration of childbirth. Actigraphy. MCTQ. PSQI. DASS.
\end{tabular}
%:::% vernacular-abstract keywords end %:::%
\end{resumoenv}
%:::% vernacular-abstract end %:::%

% -----
% Abstract in the foreign language (mandatory)
% -----

%:::% foreign-abstract begin %:::%
\begin{resumoenv}[\resumoestrangeironame]
\begin{otherlanguage*}{brazil}
%:::% foreign-abstract reference begin %:::%
Sales, A. R. V. ({\imprimirdata}). \textit{Associações entre a duração e a qualidade do sono de gestantes no terceiro trimestre com a duração do trabalho de parto} [Dissertação de Mestrado, Universidade de São Paulo].
%:::% foreign-abstract reference end %:::%

%:::% foreign-abstract body begin %:::%

O texto abaixo está relacionado ao \textbf{projeto} desta dissertação. O
resumo final só poderá ser produzido quando a pesquisa for finalizada.

Dentre as alterações biopsicossociais que ocorrem durante a gravidez
estão as mudanças no padrão do ciclo sono-vigília. Pesquisas sugerem que
há associações entre a duração e a qualidade do sono de gestantes no
período pré-natal com desfechos adversos na saúde materno infantil. O
objetivo primário deste projeto é investigar a presença/ausência de
associações significativas entre a duração e a qualidade do sono de
gestantes no terceiro trimestre com a duração do trabalho de parto. Para
isso, será realizado um estudo com 133 gestantes no terceiro trimestre
gestacional acompanhadas na casa de parto de Sapopemba, localizada no
município de São Paulo. Serão coletados dados demográficos,
antropométricos, obstétricos, actigráficos, dados relacionados ao sono e
ao estado psicológico das participantes, além de dados secundários dos
prontuários e cadernetas das gestantes. O estudo já obteve todas as
aprovações éticas para sua operação por parte das autoridades
competentes. Os resultados serão analisados pela comparação das médias
entre grupos por meio de uma análise de covariância (ANCOVA). A hipótese
básica é que uma menor qualidade e duração de sono ao longo da gestação
estão associadas a uma maior duração do trabalho de parto. Além de gerar
conhecimento para um assunto de interesse público, espera-se que os
resultados deste projeto incentivem e colaborem com a criação de novos
serviços e tecnologias de acompanhamento de gestantes.

%:::% foreign-abstract body end %:::%

%:::% foreign-abstract keywords begin %:::%
\begin{tabular}{p{3.6cm} p{12.3cm}}
  \textbf{Palavras-chaves}: &  Gestação. Gravidez. Sono. Ciclo sono-vigília. Qualidade do sono. Duração do sono. Parto. Duração do parto. Actigrafia. MCTQ. PSQI. DASS.
\end{tabular}
%:::% foreign-abstract keywords end %:::%
\end{otherlanguage*}
\end{resumoenv}
%:::% foreign-abstract end %:::%

% -----
% List of figures (optional)
% -----

%:::% list-of-figures begin %:::%
\pdfbookmark[0]{\listfigurename}{lof}
\listoffigures*
\cleardoublepage
%:::% list-of-figures end %:::%

% -----
% List of tables (optional)
% -----

%:::% list-of-tables begin %:::%
\pdfbookmark[0]{\listtablename}{lot}
\listoftables*
\cleardoublepage
%:::% list-of-tables end %:::%

% -----
% List of abbreviations and acronyms (optional)
% -----

%:::% list-of-abbreviations begin %:::%
\begin{siglas}
%:::% list-of-abbreviations body begin %:::%

\begin{description}
\item[\textsubscript{F}]
\hspace{20cm}

Subscript indicating a relation with work-free days
\item[\textsubscript{W}]
\hspace{20cm}

Subscript indicating a relation with workdays
\item[BT]
\hspace{20cm}

Local time of going to bed
\item[FD]
\hspace{20cm}

Number of work-free days per week
\item[GU]
\hspace{20cm}

Local time of getting out of bed
\item[HO]
\hspace{20cm}

Horne \& Ostberg's morningness-eveningness questionnaire (same as
\emph{MEQ})
\item[LE]
\hspace{20cm}

Light exposure
\item[LE\textsubscript{week}]
\hspace{20cm}

Average weekly light exposure
\item[MCTQ]
\hspace{20cm}

Munich ChronoType Questionnaire
\item[MCTQ\textsuperscript{PT}]
\hspace{20cm}

Portuguese version of the MCTQ
\item[MEQ]
\hspace{20cm}

Morningness-Eveningness Questionnaire
\item[MSF]
\hspace{20cm}

Midsleep on work-free days. Local time of the midpoint between sleep
onset and sleep end on work-free days
\item[MSF\textsubscript{sc}]
\hspace{20cm}

Midsleep on work-free days with a sleep correction -- MCTQ's chronotype
proxy. Same as MSF with a sleep correction (\textsubscript{SC}) made
when a possible sleep compensation related to a lack of sleep on
workdays is identified.
\item[MSW]
\hspace{20cm}

Midsleep on workdays. Local time of the midpoint between sleep onset and
sleep end on workdays.
\item[PRC]
\hspace{20cm}

Phase response curve
\item[SD]
\hspace{20cm}

Sleep duration
\item[SD\textsubscript{week}]
\hspace{20cm}

Average weekly sleep duration
\item[SE]
\hspace{20cm}

Local time of sleep end
\item[SI]
\hspace{20cm}

``Sleep inertia''. Despite the name, this abbreviation represents the
time that a person takes to get up after sleep end. It is used this way
by the MCTQ authors.
\item[SJL]
\hspace{20cm}

Absolute social jetlag
\item[SJL\textsubscript{rel}]
\hspace{20cm}

Relative social jetlag
\item[SJL\textsubscript{sc}]
\hspace{20cm}

Jankowski's sleep-corrected social jetlag
\item[SJL\textsubscript{sc-rel}]
\hspace{20cm}

Jankowski's relative sleep-corrected social jetlag
\item[Sloss\textsubscript{week}]
\hspace{20cm}

Weekly sleep loss
\item[SO]
\hspace{20cm}

Local time of sleep onset
\item[Slat]
\hspace{20cm}

Sleep latency, i.e., time (duration) to fall asleep after deciding to
sleep
\item[SPrep]
\hspace{20cm}

Local time of preparing to sleep
\item[TBT]
\hspace{20cm}

Total time in bed
\item[WD]
\hspace{20cm}

Number of workdays per week
\end{description}

%:::% list-of-abbreviations body end %:::%
\end{siglas}
%:::% list-of-abbreviations end %:::%

% -----
% List of symbols (optional)
% -----

%:::% list-of-symbols begin %:::%
\begin{simbolos}
%:::% list-of-symbols body begin %:::%

For an extensive list of chronobiology related symbols, please refer to
\textcite{aschoff1965} and \textcite{marques2012}.

\begin{description}
\item[\(\tau\)]
\hspace{20cm}

Period of a rhythm in free flow. Only revealed under constant
environmental conditions.
\item[\(T\)]
\hspace{20cm}

Zeitgeber period
\item[\(\phi\)]
\hspace{20cm}

Phase
\item[\(\Delta\phi\)]
\hspace{20cm}

Phase shift
\item[\(+\Delta\phi\)]
\hspace{20cm}

Phase advance
\item[\(-\Delta\phi\)]
\hspace{20cm}

Phase delay
\item[\(\Psi\)]
\hspace{20cm}

Phase relation
\end{description}

%:::% list-of-symbols body end %:::%
\end{simbolos}
%:::% list-of-symbols end %:::%

% -----
% Table of contents (mandatory)
% -----

%:::% table-of-contents begin %:::%
\pdfbookmark[0]{\contentsname}{toc}
\tableofcontents*
\cleardoublepage
%:::% table-of-contents end %:::%
