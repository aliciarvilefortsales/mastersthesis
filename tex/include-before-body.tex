%:::% class attribute begin/end %:::%

% -----
% Cover (mandatory)
% -----

%:::% cover begin %:::%
\imprimircapa
%:::% cover end %:::%

% -----
% Title page (mandatory)
% -----

%:::% approval-sheet begin %:::%
\imprimirfolhaderosto
%:::% approval-sheet end %:::%

% -----
% Cataloging record (mandatory)
% -----

%:::% cataloging-record begin %:::%
\begin{fichacatalografica}
%:::% cataloging-record body begin %:::%
I authorize the full or partial reproduction of this work by any conventional or electronic means for the purposes of study and research, provided that the source is cited.
\ABNTEXfontereduzida

\vfill
\begin{center}
Cataloging in publication

Library

{\imprimirescola}

\medskip
\setlength{\fboxsep}{1cm}
\fbox{\begin{minipage}[c][7.5cm]{12.5cm}
Vartanian, Daniel

\smallskip
\hspace{0.5cm} {\imprimirtitulo}  / {\imprimirautor}; supervisor, {\imprimirorientador}. {\imprimirlocal}, {\imprimirdata}.

\smallskip
{\thelastpage}p. : il

\smallskip
\hspace{0.5cm} {\imprimirtipotrabalho} (\imprimirtituloacademico) -- {\imprimirprograma}, {\imprimirescola}, {\imprimiruniversidade}, {\imprimirlocal}, {\imprimirdata}.

\smallskip
\hspace{0.5cm} {\imprimirnotadeversao}

\smallskip
\hspace{0.5cm} 1. Chronotype. 2. Entrainment. 3. Sleep. 4. Chronobiology. 5. Complex systems. I. Rodrigues Neto, Camilo, super. II. Title.

\smallskip
\hfill CDD ? .ed. ???.???
\end{minipage}}
\end{center}
%:::% cataloging-record body end %:::%
\end{fichacatalografica}
%:::% cataloging-record end %:::%

% -----
% Errata (optional)
% -----

%:::% errata begin %:::%
\begin{errata}[\errataname]
%:::% errata reference begin %:::%
\noindent Vartanian, D. ({\imprimirdata}). \textit{\imprimirtitulo} [{\imprimirtipodetituloacademico}'s {\imprimirtipotrabalho}, {\imprimiruniversidade}].
%:::% errata reference end %:::%
\smallskip
%:::% errata body begin %:::%

This is the preliminary version of this thesis (version \textless1.0.0).
Any required corrections will be listed here upon approval.

%:::% errata body end %:::%
\end{errata}
%:::% errata end %:::%

% -----
% Approval sheet (mandatory)
% -----

%:::% approval-sheet begin %:::%
\begin{folhadeaprovacao}[\folhadeaprovacaoname]
%:::% approval-sheet body begin %:::%
{\imprimirtipotrabalho} by {\imprimirautor}, under the title \textbf{\imprimirtitulo}, presented to the {\imprimirescola} at the {\imprimiruniversidade}, as a requirement for the degree of {\imprimirtituloacademico} by the {\imprimirprograma}, in the concentration area of {\imprimirareadeconcentracao}.

\vspace{\hugeskipamount}
Approved on \_\_\_\_\_\_\_\_\_\_\_\_\_\_\_\_\_\_\_\_ , \_\_\_\_\_\_\_\_\_\_ .

\vspace{\hugeskipamount}
\begin{center}
  Examination committee
\end{center}

\vspace{\smallskipamount}
Committee chair:

\vspace{\tinyskipamount}
\begingroup

\AtBeginEnvironment{tabular}{
  \normalsize
  \renewcommand{\arraystretch}{2}
}

\setlength{\arrayrulewidth}{0pt}
\setlength{\tabcolsep}{0cm}
\begin{tabular}{m{2cm} P{14cm}}
  Prof. Dr. & \_\_\_\_\_\_\_\_\_\_\_\_\_\_\_\_\_\_\_\_\_\_\_\_\_\_\_\_\_\_\_\_\_\_\_\_\_\_\_\_\_\_\_\_\_\_\_\_\_\_\_\_\_\_\_ \\
  Institution & \_\_\_\_\_\_\_\_\_\_\_\_\_\_\_\_\_\_\_\_\_\_\_\_\_\_\_\_\_\_\_\_\_\_\_\_\_\_\_\_\_\_\_\_\_\_\_\_\_\_\_\_\_\_\_ \\
\end{tabular}

\vspace{\bigskipamount}
Examiners:

\vspace{\tinyskipamount}
\begin{tabular}{m{2cm} P{14cm}}
  Prof. Dr. & \_\_\_\_\_\_\_\_\_\_\_\_\_\_\_\_\_\_\_\_\_\_\_\_\_\_\_\_\_\_\_\_\_\_\_\_\_\_\_\_\_\_\_\_\_\_\_\_\_\_\_\_\_\_\_ \\
  Institution & \_\_\_\_\_\_\_\_\_\_\_\_\_\_\_\_\_\_\_\_\_\_\_\_\_\_\_\_\_\_\_\_\_\_\_\_\_\_\_\_\_\_\_\_\_\_\_\_\_\_\_\_\_\_\_ \\
  Evaluation & \_\_\_\_\_\_\_\_\_\_\_\_\_\_\_\_\_\_\_\_\_\_\_\_\_\_\_\_\_\_\_\_\_\_\_\_\_\_\_\_\_\_\_\_\_\_\_\_\_\_\_\_\_\_\_ \\
\end{tabular}

\vspace{\smallskipamount}
\begin{tabular}{m{2cm} P{14cm}}
  Prof. Dr. & \_\_\_\_\_\_\_\_\_\_\_\_\_\_\_\_\_\_\_\_\_\_\_\_\_\_\_\_\_\_\_\_\_\_\_\_\_\_\_\_\_\_\_\_\_\_\_\_\_\_\_\_\_\_\_ \\
  Institution & \_\_\_\_\_\_\_\_\_\_\_\_\_\_\_\_\_\_\_\_\_\_\_\_\_\_\_\_\_\_\_\_\_\_\_\_\_\_\_\_\_\_\_\_\_\_\_\_\_\_\_\_\_\_\_ \\
  Evaluation & \_\_\_\_\_\_\_\_\_\_\_\_\_\_\_\_\_\_\_\_\_\_\_\_\_\_\_\_\_\_\_\_\_\_\_\_\_\_\_\_\_\_\_\_\_\_\_\_\_\_\_\_\_\_\_ \\
\end{tabular}

\vspace{\smallskipamount}
\begin{tabular}{m{2cm} P{14cm}}
  Prof. Dr. & \_\_\_\_\_\_\_\_\_\_\_\_\_\_\_\_\_\_\_\_\_\_\_\_\_\_\_\_\_\_\_\_\_\_\_\_\_\_\_\_\_\_\_\_\_\_\_\_\_\_\_\_\_\_\_ \\
  Institution & \_\_\_\_\_\_\_\_\_\_\_\_\_\_\_\_\_\_\_\_\_\_\_\_\_\_\_\_\_\_\_\_\_\_\_\_\_\_\_\_\_\_\_\_\_\_\_\_\_\_\_\_\_\_\_ \\
  Evaluation & \_\_\_\_\_\_\_\_\_\_\_\_\_\_\_\_\_\_\_\_\_\_\_\_\_\_\_\_\_\_\_\_\_\_\_\_\_\_\_\_\_\_\_\_\_\_\_\_\_\_\_\_\_\_\_ \\
\end{tabular}
\endgroup
%:::% approval-sheet body end %:::%
\end{folhadeaprovacao}
%:::% approval-sheet end %:::%

% -----
% Inscription (optional)
% -----

%:::% inscription begin %:::%
\begin{dedicatoria}[] % \dedicatorianame | Keep #1 empty.
\vspace*{\fill} % Don't change it.
\centering
%:::% inscription body begin %:::%
\textit{I dedicate this work to the skeptics, the radicals, the ignorant, the uncivilized, the subversives, the wild dogs, the irreducibles, the irreconcilables. To the true engines of change. To the destabilizers, who possess equal or greater importance than the stabilizers. To those who act on principle, even knowing that there is no ultimate reward or any meaning in life.}
%:::% inscription body end %:::%
\vspace*{\fill} % Don't change it.
\vspace{4.5cm}
% \vspace{13cm}
\end{dedicatoria}
%:::% inscription end %:::%

% -----
% Acknowledgments (optional)
% -----

%:::% acknowledgments begin %:::%
\begin{agradecimentos}[\agradecimentosname]
  %:::% acknowledgments body begin %:::%

I would like to acknowledge and express my gratitude to the following
persons and organizations:

Salete Perroni (Sal), my partner in life and in the fight for a better
world.

My Mother, for her unconditional love.

My sister and my brother, for their love and companionship in life.

My friends in science,
\href{https://orcid.org/0000-0003-0004-4487}{Alicia Rafaelly Vilefort
Sales}, \href{https://orcid.org/0000-0002-8222-5240}{Juliana Viana
Mendes}, and \href{https://orcid.org/0000-0002-9283-9967}{Maria Augusta
Medeiros de Andrade}.

My friend and Professor
\href{https://orcid.org/0000-0002-1164-2055}{Humberto Miguel Garay
Malpartida}, for his support; for his principles; and for his integrity,
which was demonstrated when the need arose.

Professor \href{https://orcid.org/0000-0001-6783-6695}{Camilo Rodrigues
Neto}, for introducing me to and teaching me about the science of
complex systems since 2012; for supervising my dissertation; for the
patience and the virtue in taking on and mediating the process of
transitioning my master's supervision after the breakdown of relations
with my former supervisor.

Professor \href{https://orcid.org/0000-0003-2916-4415}{Carlos Molina
Mendes}, for his speed, impartiality, patience, and virtuous approach in
mediating the process of transitioning of my master's supervision.

My fellow friends: Alex Azevedo Martins; Amanda Moreira; Augusto Amado,
Carina (Cacau) Prado; Ítalo Alves Bezerra do Nascimento; Júlia Mafra;
Letícia Nery de Figueiredo; Marcelo Ricardo Fernandes Roschel; Reginaldo
Noveli; Sílvia Capelanes; and Vanessa Simon Silva.

\href{https://lula.com.br/}{President Lula} (Yes!), who saved Brazil
from fascism and approved the long-overdue adjustments to graduate
scholarships.

The local student movements, which truly support their category.

The \href{https://prip.usp.br/apoio-estudantil/}{Support Program for
Student Permanence and Education (PAPFE)} of USP, which enabled me to
get this far.

The \href{https://www.gov.br/capes/}{Coordination for the Improvement of
Higher Education Personnel (CAPES)}, for funding this work and enabling
my presence in graduate studies (Grant number: 88887.703720/2022-00).

  %:::% acknowledgments body end %:::%
\end{agradecimentos}
%:::% acknowledgments end %:::%

% -----
% Epigraph (optional)
% -----

%:::% epigraph begin %:::%
\begin{epigrafe}[] % \epigraphname | Keep #1 empty.
\vspace*{\fill} % Don't change it.
\begin{flushright}
%:::% epigraph body begin %:::%
\textit{Nullius in verba}\footnotemark{}

\footnotetext{
  The Royal Society. (n.d.). \textit{History of the Royal Society}. \href{https://royalsociety.org/about-us/history/}{https://royalsociety.org/about-us/history/}
}
%:::% epigraph body end %:::%
\end{flushright}
\end{epigrafe}
%:::% epigraph end %:::%

% -----
% Abstract in the vernacular language (mandatory)
% -----

%:::% vernacular-abstract begin %:::%
\begin{resumoenv}[\resumoname]
 %:::% vernacular-abstract reference begin %:::%
Vartanian, D. ({\imprimirdata}). \textit{\imprimirtitulo} [{\imprimirtipodetituloacademico}'s {\imprimirtipotrabalho}, {\imprimiruniversidade}].
%:::% vernacular-abstract reference end %:::%

%:::% vernacular-abstract body begin %:::%

The text below is related to the \textbf{project} of this thesis. The
final abstract can only be produced when the research is completed.

Theories related to sleep and circadian rhythms are already
well-established in science. However, it is necessary to verify and test
these same theories in more extensive samples to obtain a more accurate
picture of the ecology of sleep and temporal phenotypes. This thesis
undertakes this commitment, with the aim of mapping the expression of
sleep-wake cycles and circadian phenotypes in the Brazilian adult
population and investigating the hypothesis that latitude is associated
with circadian rhythm regulation. The latitude hypothesis is based on
the idea that regions located at latitudes near the poles have, on
average, a lower annual incidence of sunlight compared to regions near
the equator (latitude 0°). Therefore, it is deduced that regions near
the equator have a stronger solar zeitgeber, which, according to
chronobiology theories, could lead to a greater propensity for the
synchronization of circadian rhythms in these populations, reducing the
amplitude and diversity of circadian phenotypes. This would also give
these populations a morning characteristic when compared to populations
living far from the equator. To achieve the aforementioned objectives,
this thesis project will rely on a data sample of sleep-wake cycle
expression in the Brazilian population, composed of \(120,265\) subjects
covering all Brazilian states. This data was obtained in 2017 and is
based on the Munich ChronoType Questionnaire (MCTQ), a widely validated
questionnaire used to measure circadian phenotypes based on the
sleep-wake cycle expression of individuals in their last four weeks. The
results will contribute to the validation of chronobiology theories and
will generate greater knowledge about the regulation of circadian
rhythms and sleep-wake cycles in the Brazilian population.

%:::% vernacular-abstract body end %:::%

%:::% vernacular-abstract keywords begin %:::%
\begin{tabular}{p{2.3cm} p{13.6cm}}
  \textbf{Keywords}: & Chronobiology. Biological rhythms. Chronotype. Circadian phenotype. Sleep. Complex systems. Entrainment. Latitude. Ecology. MCTQ.
\end{tabular}
%:::% vernacular-abstract keywords end %:::%
\end{resumoenv}
%:::% vernacular-abstract end %:::%

% -----
% Abstract in the foreign language (mandatory)
% -----

%:::% foreign-abstract begin %:::%
\begin{resumoenv}[\resumoestrangeironame]
\begin{otherlanguage*}{brazil}
%:::% foreign-abstract reference begin %:::%
Vartanian, D. ({\imprimirdata}). \textit{Ecologia do sono e de fenótipos circadianos da população brasileira} [Dissertação de Mestrado, Universidade de São Paulo].
%:::% foreign-abstract reference end %:::%

%:::% foreign-abstract body begin %:::%

O texto abaixo está relacionado ao \textbf{projeto} desta dissertação. O
resumo final só poderá ser produzido quando a pesquisa for finalizada.

Teorias relacionadas ao sono e aos ritmos circadianos já estão bem
consolidadas na ciência. No entanto, é necessário verificar e testar
essas mesmas teorias em amostras mais abrangentes para obter um retrato
mais preciso da ecologia do sono e dos fenótipos temporais. Esta
dissertação assume esse compromisso, tendo como objetivo mapear a
expressão dos ciclos de sono-vigília e dos fenótipos circadianos da
população adulta brasileira e investigar a hipótese de que a latitude
está associada à regulação do ritmo circadiano. A hipótese da latitude
se fundamenta na ideia de que regiões localizadas em latitudes próximas
aos polos apresentam, em média, uma menor incidência de luz solar anual
quando comparadas com regiões próximas da linha do equador (latitude
0°). Dessa forma, deduz-se que as regiões próximas ao equador apresentam
um zeitgeber solar mais forte, o que, de acordo com as teorias da
cronobiologia, pode gerar uma maior propensão à sincronização dos ritmos
circadianos dessas populações, reduzindo a amplitude e a diversidade de
fenótipos circadianos. Isso também daria a essas populações uma
característica matutina quando comparadas com populações que vivem
distantes da linha do equador. Para atingir os objetivos mencionados, o
projeto irá contar com uma amostra de dados da expressão do ciclo
sono-vigília da população brasileira composta por \(120.265\) indivíduos
que abrange todos os estados brasileiros. Essa amostra de dados foi
obtida no ano de 2017 e se baseia no Munich ChronoType Questionnaire
(MCTQ), um questionário amplamente validado e utilizado para mensurar
fenótipos circadianos a partir da expressão do ciclo sono-vigília de
indivíduos em suas últimas quatro semanas. Os resultados irão contribuir
com a validação de teorias da cronobiologia e gerar conhecimento sobre a
regulação do ritmo circadiano e dos ciclos de sono-vigília da população
brasileira.

%:::% foreign-abstract body end %:::%

%:::% foreign-abstract keywords begin %:::%
\begin{tabular}{p{3.6cm} p{12.3cm}}
  \textbf{Palavras-chaves}: &  Cronobiologia. Ritmos biológicos. Cronotipo. Fenótipo circadiano. Sono. Sistemas complexos. Entrainment. Latitude. Ecologia. MCTQ.
\end{tabular}
%:::% foreign-abstract keywords end %:::%
\end{otherlanguage*}
\end{resumoenv}
%:::% foreign-abstract end %:::%

% -----
% List of figures (optional)
% -----

%:::% list-of-figures begin %:::%
\pdfbookmark[0]{\listfigurename}{lof}
\listoffigures*
\cleardoublepage
%:::% list-of-figures end %:::%

% -----
% List of tables (optional)
% -----

%:::% list-of-tables begin %:::%
\pdfbookmark[0]{\listtablename}{lot}
\listoftables*
\cleardoublepage
%:::% list-of-tables end %:::%

% -----
% List of abbreviations and acronyms (optional)
% -----

%:::% list-of-abbreviations begin %:::%
\begin{siglas}
%:::% list-of-abbreviations body begin %:::%

\begin{description}
\item[\textsubscript{F}]
\hspace{20cm}

Subscript indicating a relation with work-free days
\item[\textsubscript{W}]
\hspace{20cm}

Subscript indicating a relation with workdays
\item[BT]
\hspace{20cm}

Local time of going to bed
\item[FD]
\hspace{20cm}

Number of work-free days per week
\item[GU]
\hspace{20cm}

Local time of getting out of bed
\item[HO]
\hspace{20cm}

Horne \& Ostberg's morningness-eveningness questionnaire (same as
\emph{MEQ})
\item[LE]
\hspace{20cm}

Light exposure
\item[LE\textsubscript{week}]
\hspace{20cm}

Average weekly light exposure
\item[MCTQ]
\hspace{20cm}

Munich ChronoType Questionnaire
\item[MCTQ\textsuperscript{PT}]
\hspace{20cm}

Portuguese version of the MCTQ
\item[MEQ]
\hspace{20cm}

Morningness-Eveningness Questionnaire
\item[MSF]
\hspace{20cm}

Midsleep on work-free days. Local time of the midpoint between sleep
onset and sleep end on work-free days
\item[MSF\textsubscript{sc}]
\hspace{20cm}

Midsleep on work-free days with a sleep correction -- MCTQ's chronotype
proxy. Same as MSF with a sleep correction (\textsubscript{SC}) made
when a possible sleep compensation related to a lack of sleep on
workdays is identified.
\item[MSW]
\hspace{20cm}

Midsleep on workdays. Local time of the midpoint between sleep onset and
sleep end on workdays.
\item[PRC]
\hspace{20cm}

Phase response curve
\item[SD]
\hspace{20cm}

Sleep duration
\item[SD\textsubscript{week}]
\hspace{20cm}

Average weekly sleep duration
\item[SE]
\hspace{20cm}

Local time of sleep end
\item[SI]
\hspace{20cm}

``Sleep inertia''. Despite the name, this abbreviation represents the
time that a person takes to get up after sleep end. It is used this way
by the MCTQ authors.
\item[SJL]
\hspace{20cm}

Absolute social jetlag
\item[SJL\textsubscript{rel}]
\hspace{20cm}

Relative social jetlag
\item[SJL\textsubscript{sc}]
\hspace{20cm}

Jankowski's sleep-corrected social jetlag
\item[SJL\textsubscript{sc-rel}]
\hspace{20cm}

Jankowski's relative sleep-corrected social jetlag
\item[Sloss\textsubscript{week}]
\hspace{20cm}

Weekly sleep loss
\item[SO]
\hspace{20cm}

Local time of sleep onset
\item[Slat]
\hspace{20cm}

Sleep latency, i.e., time (duration) to fall asleep after deciding to
sleep
\item[SPrep]
\hspace{20cm}

Local time of preparing to sleep
\item[TBT]
\hspace{20cm}

Total time in bed
\item[WD]
\hspace{20cm}

Number of workdays per week
\end{description}

%:::% list-of-abbreviations body end %:::%
\end{siglas}
%:::% list-of-abbreviations end %:::%

% -----
% List of symbols (optional)
% -----

%:::% list-of-symbols begin %:::%
\begin{simbolos}
%:::% list-of-symbols body begin %:::%

For an extensive list of chronobiology related symbols, please refer to
\textcite{aschoff1965} and \textcite{marques2012}.

\begin{description}
\item[\(\tau\)]
\hspace{20cm}

Period of a rhythm in free flow. Only revealed under constant
environmental conditions.
\item[\(T\)]
\hspace{20cm}

Zeitgeber period
\item[\(\phi\)]
\hspace{20cm}

Phase
\item[\(\Delta\phi\)]
\hspace{20cm}

Phase shift
\item[\(+\Delta\phi\)]
\hspace{20cm}

Phase advance
\item[\(-\Delta\phi\)]
\hspace{20cm}

Phase delay
\item[\(\Psi\)]
\hspace{20cm}

Phase relation
\end{description}

%:::% list-of-symbols body end %:::%
\end{simbolos}
%:::% list-of-symbols end %:::%

% -----
% Table of contents (mandatory)
% -----

%:::% table-of-contents begin %:::%
\pdfbookmark[0]{\contentsname}{toc}
\tableofcontents*
\cleardoublepage
%:::% table-of-contents end %:::%
